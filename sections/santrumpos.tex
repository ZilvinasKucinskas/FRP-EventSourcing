\begin{itemize}

	\item \textbf{CQRS} (angl. Command Query Responsibility Segregation) – komandų-užklausų atsakomybių atskyrimas.

	\item \textbf{DDD} (angl. Domain-Driven Design) – būdas kurti programinę įrangą, skirtą spręsti sudėtingus uždavinius, bei apjungti realizaciją kartu su augančiu domeno modeliu.

	\item \textbf{NoSQL} – duomenų bazė, skirta architektūriniams modeliams, kuriems nereikia palaikyti stiprios darnos principo, kuris naudojamas reliacinėse duomenų bazėse. Tai įgalina horizontalų išplečiamumą bei aukštesnį prieinamumą.

	\item \textbf{ORM} (angl. Object-Relational Mapping) – programavimo technika duomenų konvertavimui tarp nesuderinamų sistemų tipų, naudojama objektinio programavimo kalbose. Pavyzdys: JAVA programavimo kalba bei Oracle duomenų bazė.

\end{itemize}