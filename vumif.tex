\documentclass[12pt, a4paper, lithuanian]{article}

\usepackage[utf8x]{inputenc}
\def\LTfontencoding{L7x}
\usepackage[\LTfontencoding]{fontenc}
\usepackage[lithuanian]{babel}
\usepackage{amsmath}
\usepackage{float}

\usepackage{VUMIF}
\usepackage{listings}
\usepackage{graphicx}

\usepackage{titlesec}

\AtBeginDocument{\setlength\abovedisplayskip{0pt}}

\setcounter{secnumdepth}{4}

\titleformat{\paragraph}
{\normalfont\normalsize\bfseries}{\theparagraph}{1em}{}
\titlespacing*{\paragraph}
{0pt}{3.25ex plus 1ex minus .2ex}{1.5ex plus .2ex}

% "define" Scala
\lstdefinelanguage{scala}{
  morekeywords={abstract,case,catch,class,def,%
    do,else,extends,false,final,finally,%
    for,if,implicit,import,match,mixin,%
    new,null,object,override,package,%
    private,protected,requires,return,sealed,%
    super,this,throw,trait,true,try,%
    type,val,var,while,with,yield},
  otherkeywords={=>,<-,<\%,<:,>:,\#,@},
  sensitive=true,
  morecomment=[l]{//},
  morecomment=[n]{/*}{*/},
  morestring=[b]",
  morestring=[b]',
  morestring=[b]""
}

\usepackage{color}
\definecolor{dkgreen}{rgb}{0,0.6,0}
\definecolor{gray}{rgb}{0.5,0.5,0.5}
\definecolor{mauve}{rgb}{0.58,0,0.82}


% Default settings for code listings
\lstset{frame=tb,
  language=scala,
  captionpos=b,
  aboveskip=3mm,
  belowskip=3mm,
  showstringspaces=false,
  columns=flexible,
  basicstyle={\small\ttfamily},
  numbers=none,
  numberstyle=\tiny\color{gray},
  %keywordstyle=\color{blue},
  %commentstyle=\color{dkgreen},
  %stringstyle=\color{mauve},
  frame=single,
  breaklines=true,
  breakatwhitespace=true
  tabsize=3
}

\renewcommand{\lstlistingname}{Kodo pavyzdys}

% \usepackage[mathcsdepttitle]{VUMIF} % --- matematinės informatikos katedros
%     titulinio puslapio formatavimas

% Titulinio puslapio reikalai
\vumifdept{Informatikos katedra}
\vumifpaper{Mokslo tiriamasis darbas}
\title{Funkcinio-reaktyvaus programavimo taikymas įvykių kaupimo sistemose}
\engtitle{Functional Reactive Programming in Event Sourcing Systems}
\author{
    1 kurso 12 grupės studentas \\
    Žilvinas Kučinskas
}

\supervisor{Viačeslav Pozdniakov}
\reviewer{prof. Rimantas Vaicekauskas}
\date{Vilnius \\ 2014}

\begin{document}

\maketitle

\tableofcontents

\section{Magistro darbo objekto apžvalga bei tyrimo problemos aprašymas}
\subsection{Tyrimo objektas}

    Tyrimo objektas yra funkcinio-reaktyvaus programavimo bei įvykių kaupimo principai.

\subsection{Darbo tikslai ir uždaviniai}

    Darbo tikslas yra pritaikyti funkcinio-reaktyvaus programavimo principus įvykių kaupimo sistemose taip, jog būtų išpildyti šie reikalavimai:

\begin{itemize}

    \item įvykių kaupimo sistemos skaitymo modelis būtų be būsenos;

    \item įvykių kaupimo sistemos skaitymo modelis būtų kuriamas tik per įvykių kompoziciją;

    \item įvykių kaupimo sistemos skaitymo modelio programinis kodas būtų griežtai tipizuotas.

\end{itemize}

    Siekiant šio tikslo, turi būti išspręsti šie uždaviniai:

\begin{itemize}
        \item įrodyti, kad funkcinį-reaktyvų programavimą įmanoma taikyti įvykių
            kaupimo sistemose;
        \item sukurti konkretizuotą kalbą (angl. domain specific language), apjungiančią funkcinio-reaktyvaus programavimo
            bei įvykių kaupimo principus;
        \item aprašyti konkretizuotuos kalbos kūrimo metodiką, apibrėžti gautų rezultatų apribojimus, suformuluoti iškilusias problemas bei paaiškinti jų priežastis.
\end{itemize}

\subsection{Tyrimo aktualumas}

    Funkcinis reaktyvus programavimas integruoja laiko tėkmę bei sudėtinius įvykius į funkcinį programavimą. Šis principas suteikia elegantišką būdą išreikšti skaičiavimus interaktyvių animacijų, robotikos, kompiuterinio vaizdavimo, vartotojo sąsajos ir modeliavimo srityse \cite[p. 4]{ELM:FRP}. Pagrindinės funkcinio reaktyvaus programavimo sąvokos:

\begin{itemize}
        \item signalai arba elgsena - reikšmės, besikeičiančios bėgant laikui;
        \item įvykiai - momentinių reikšmių kolekcijos arba laiko-reikšmės poros.
\end{itemize}

    Funkcinis-reaktyvus programavimas įgalina apsirašyti elgseną deklaratyviai \cite[p.1]{ElliottHudak97:Fran}. Elgsena ir įvykiai gali būti komponuojami kartu, išreikšti vienas per kitą. Funkcinis reaktyvus programavimas apibrėžia kaip signalai arba elgsena reaguoja į įvykius. \cite[p. 1]{Survey} Šį principą galima iliustruoti pavyzdžiu. Tarkime turime Excel lapą, kuriame yra trys laukai: darbuotojo pradirbtos valandos, valandinis užmokestis bei formulė, kuri paskaičiuoja konkretų darbuotojo atlyginimą. Darbuotojui pradirbus daugiau valandų, atnaujinamas pradirbtų valandų skaičius. Kartu atsinaujina ir pačios formulės reikšmė, tai yra konkretus užmokestis pakinta. Šiuo atveju įvykus reikšmės atnaujinimo įvykiui, nuo jos priklausomos formulės taipogi atsinaujina arba tam tikras įvykis iššaukia elgseną sistemoje.

    Įvykių kaupimo principo esmė – objektas yra atvaizduojamas kaip įvykių seka. Kaip pavyzdį tai galima parodyti remiantis banko sąskaita. Tarkime vartotojas, banko klientas, turi 100 litų sąskaitos balansą. Tarkime vartotojas nusipirko prekę už 20 litų, tada įnešė į savo sąskaitą 15 litų ir galiausiai nusipirko tam tikrą paslaugą už 30 litų. Akivaizdu, jog turint šią įvykių seką, galima atvaizduoti dabartinę objekto būseną - tai yra 65 litai vartotojo sąskaitoje. Įvykių kaupimo principas užtikrina, jog visi būsenos pasikeitimai yra saugomi įvykių žurnale kaip įvykų seka \cite{vernon2013implementing}. Įvykių kaupimo principui yra būdinga, jog įvykių negalimą ištrinti bei atnaujinti, duomenys yra nekeičiami, dėl to įvykių žurnalas yra sistemos gyvavimo istorija (tiesos šaltinis). Tačiau toks modelis turi ir trūkumų. Jis nėra pritaikytas patogiam užklausų rašymui. Iš įvykių srautų yra kuriamos projekcijos, skirtos konkretiems sistemoms poreikiams, pavyzdžiui: paieškai, klasifikacijai ar ataskaitų ruošimui.

    Pritaikius funkcinį-reaktyvų programavimą įvykių kaupimo principu paremtose sistemose būtų galima modeliuoti ne tik momentinius įvykius, tačiau turėti ir jų istoriją. Yra poreikis sukurti konkretizuotą kalbą (angl. domain specific language), kuri įgalintų paslėpti įvykių žurnalą (arba duomenų saugyklą). Pastarosios naudotojas galėtų orientuotis į pačią sprendžiamos srities problemą, nekreipdamas dėmesio į žemesnio lygio realizacijos detales. Šiuo atveju vbūtų galima deklaratyviai (ką kažkuri programos dalis turi daryti) apsirašyti elgseną, nutikus įvykiui, kartu su imperatyviomis(instrukcijos, kurios aprašo, kaip programos dalys atlieka savo užduotis) struktūromis.

\subsection{Pritaikymo pavyzdys}

    Tarkime turime domeno sritį - bankininkystė. Turime įvykių srautą - vartotojų sukūrimas. Naudojant įsivaizduojamą Scala API galima sukurti vartotojų paieškos puslapį pagal vardą ir pavardę. Demonstacija pateikta \ref{creation} kodo pavyzdyje.

\begin{lstlisting}[caption=- vartotojų paieškos puslapis naudojant įsivaizduojamą Scala API, label=creation]
// stream model
case class CustomerCreate (val name: String, val surname: String, val personalNum: String)

val es = EventSourceConnection(url)
val createStream = Stream(es, "customerCreate")

class case CustomerModel(val name: String, val surname: String, val personalNum: String)  extends ViewableModel

trait CustomerArgModel extends Arg2Model[String,String]{
  val name: Option[String]
  val surname: Option[String]
}

//args are passed on form view/post
val customerView = View(args: CustomerArgModel).foldLeft(
  (acc,event) => event match {
    case CustomerCreate(name, surname, personalNum) =>
      for {
        newName <- name
        newSurname <- surname
        newPersNum <- personalNum
        if (args.name == name && args.surname == surname)
      } yield CustomerModel(newName, newSur, newPersNum)
  }
)
// getting data for all Kucinskai
val specificData = customerView(None, Some("Kucinskas")): Option[List[CustomerModel]]
\end{lstlisting}

    Šiuo atveju veiksmai peržiūrint duomenis yra sumaišomi kartu su veiksmais gaunant įvykius. Verta pastebėti, jog lokali duomenų saugykla nebuvo paminėta arba apibrėžta. Pastaroji gali būti sugeneruota bei valdoma automatiškai.

    Antruoju pavyzdiniu atveju turimas vartotojo balanso(įplaukiančios/išplaukiančios lėšos) įvykių srautas ir norima gauti vartotojo, kurio asmens kodas yra \emph{39008226547}, einamosios savaitės sąskaitos balansą. Demonstacija pateikta \ref{balance} kodo pavyzdyje.

\begin{lstlisting}[caption=- vartotojo einamosios sąskaitos balansas naudojant įsivaizduojamą Scala API, label=balance]
val duration = 1.weeks
val personalNum = "39008226547"
val balanceStream = Stream(es, "customerBalance")
val notOlderThanOneWeek =
    for {
        event <- balanceStream
        filtered <- event.filter(_.personalNum == personalNum
            && (DateTime.now - _.timeStamp) >= duration)
    } yield filtered
val sum = notOlderThanOneWeek.toList.sum
\end{lstlisting}

    Šiuo atveju \emph{event} kintamasis yra galimai įvykių srauto monada (terminas vartojamas funkciniame programavime, kilęs iš kategorijų teorijos ir turi savas taisykles), o \emph{filtered} kintamasis - duomenų saugyklos monada. Bendruoju atveju skirtingos monados tarpusavyje nesiderina \cite{DBLP:conf/fp/KingW92}. Dėl to reikia išsiaiškinti - ar yra įmanoma ir kaip šias monadas suderinti.

\subsection{Tyrimo metodika}

    Darbo tikslui pasiekti tiriamojoje dalyje bus pasirinkta konkreti funkcinė programavimo kalba (pvz.: Haskell, Scala) bei aprašoma kūrimo metodika.

\subsection{Laukiami rezultatai}

    Magistrinio darbo metu planuojama išnagrinėti funkcinio-reaktyvaus programavimo ir įvykio kaupimo principus, įrodyti, jog šie principai gali būti panaudoti kartu bei suderinti, sukurti konkretizuotą kalbą (angl. domain specific language), apjungiančią šiuos principus, bei aprašyti kūrimo eigos metodiką, apibrėžti gautus rezultatus, suformuluoti apribojimus, iškilusias problemas bei paaiškinti jų priežastis.

\section{Literatūros analizė}
% \subsection{Funkcinis-reaktyvus programavimas}

\subsubsection{Įvadas}

Anot \cite{Survey}, funkcinis-reaktyvus programavimas, toliau dažnai vadinamas tiesiog FRP, yra būdas modeliuoti reaktyvų - besikeičiančius laiko tekmėje bei reaguojančius į išorinį stimulą - elgesį visiškai funkcinėse programavimo kalbose. FRP leidžia deklaratyviu ir paprastu būdu modeliuoti sistemas, kurios turi reaguoti į duomenis bėgant laikui.

\subsubsection{Pagrindinis tiklas}

Pagrindinis funkcinio-reaktyvaus programavimo tikslas \cite{Survey}:

\begin{itemize}

	\item saugus programavimas - kompiliatorius turi kiek įmanoma patikrinti programų korektiškumą;

	\item efektyvus programavimas - programos turėtų veikti realiu laiku, todėl efektyvios ir optimizuotos operacijos yra būtinos;

	\item komponavimas - FRP leidžia kurti programas iš smulkesnių programų, o ne orientuotą į problemą, vientisą kodą.

\end{itemize}

\subsubsection{Sąvokos}

Pagrindinės FRP sąvokos yra:

\begin{itemize}

	\item signalai arba elgsena - besikeičiančios laike reikšmės;

	\item įvykiai - kolekcija momentinių reikšmių arba laiko-reikšmės poros.

\end{itemize}

FRP pasiekia reaktyvumą naudodamas konstrukcijas, kurios tiksliai apibrėžia kaip signalai arba elgsena pasikeičia reaguodami į įvykius. Tai yra pagrindinis būdas išreiškiant bei realizuojant elgseną. Kitu būdu, elgsena gali būti laiko semantinės funkcijos\footnote{http://msdl.cs.mcgill.ca/people/tfeng/docs/as/node5.html}, kuriose laikas yra pakeičiamas kilus įvykiui \cite{Nilsson:2002:FRP:581690.581695}.

\subsubsection{Sąvybės}

Anot anksčiausios funkcinio-reaktyvaus programavimo formuluotės \cite{ElliottHudak97:Fran}, pagrindinės sąvybės, kuriomis pasižymi FRP:

\begin{itemize}

	\item elgsenos arba signalų modeliavimas bėgant laikui,

	\item įvykių, kurie turi baigtinį skaičių atsitikimų daugelyje laiko taškų, modeliavimas,

	\item perjungimas (angl. switching) - sistema gali pasikeisti dėl atsitikusių įvykių,

	\item analizės detalių, tokių kaip reaktyvaus modelio įvykių ėmimo dažnis, atskyrimas.

\end{itemize}

\subsubsection{Įvykių srautas}

Pagal \cite{Bass:2007:Mythbusters}, įvykių srautas yra eilė pagal laiką surikiuotų įvykių, pavyzdžiui akcijų rinkos srautas.

Įvykių srautas kaip duomenų srauto tipas formaliai atrodo kaip pora (s, t), kur s yra seka surikiuotų sąrašo įvykių, o t yra seka laiko intervalų ir kiekvienas intervalas yra netuščias.

Tokio duomenų srauto pavyzdžiai gali būti:

\begin{itemize}

	\item akcijų kursas,

	\item paspaudimų srautas,

	\item tinklo srautas,

	\item GPS\footnote{http://en.wikipedia.org/wiki/Global\_Positioning\_System} duomenys.

\end{itemize}

Įvykių srauto apdorojimas pagal atsitikimo laiką turi privalumų:

\begin{itemize}

	\item įvykių apdorojimo algoritmai naudoja mažai atminties, nes jiems nereikia prisiminti daug įvykių;

	\item algoritmai gali būti labai greiti;

	\item gavus įvykį, skaičiavimai atliekami iškart, todėl galima perduoti rezultatą kitam skaičiavimui ir pamiršti įvykį.

\end{itemize}

Įvykių srauto apdorojimas labiau akcentuoja didelio našumo duomenų gavimą ir matematinių algoritmų pritaikymą įvykių duomenims. Taip pat įvykių srautai įprastai pritaikomi konkrečiai sistemai ar organizacijai.

\subsubsection{Įvykių srautas funkciniame programavime}

Vietoje įvykių srauto, galima naudoti klausytojo projektavimo šabloną \cite{WhiteboardPattern}, tačiau ilgame programinės įrangos kūrimo gyvavimo cikle įvykių srauto naudojimas palengvina kai kuriuos dalykus. Pavyzdžiui, imperatyvaus programavimo atžvilgiu, viename metode turime klausytoją, kuris reaguoją į situaciją \textit{A}, tačiau iškilus situacijai \textit{B}, šis klausytojas yra pašalinamas. Šiuo atveju programinis kodas, kuris valdo klausytojo gyvavimo ciklą yra įsipynęs keliose skirtingose kodo vietose, ko pasekoje tai reiškia, jog yra sunkiau palaikyti, stebėti, keisti bei suprasti šias vietas. Įvykių srautas yra pirmosios klasės reikšmės, todėl jį galima abstrahuoti \cite{EPFL-REPORT-176887}. Abstrahuojant klausytojų gyvavimo ciklo valdymo idėją tampa lengviau programuoti pagal tai kas turėtų nutikti, o ne pagal tai, ką kompiuteris turėtų daryti toliau. Tačiau turbūt didesnis įvykių srauto pranašumas yra tai, jog pastarasis gali būti transformuojamas ir komponuojamas. Toliau bus aprašomi operacijos arba veiksmai bei sutrumpinti išeities kodo pavyzdžiai, kuriuos galima atlikti su įvykių srautu remiantis Reactive Web\footnote{http://scalareactive.org} karkasu, skirtu Scala\footnote{http://www.scala-lang.org} programavimo kalbai.

Pirmas veiksmas - įvykio srauto sukūrimas. Jis pavaizduotas \ref{creation} kodo pavyzdyje.  Pirmiausia sukuriamas įvykių šaltinis, kuris po to priskiriamas įvykių srautui.

\begin{lstlisting}[caption=- įvykių srauto sukūrimas, label=creation]
	val eventSource = new EventSource[String]{}
	scheduleTask(10000) {
		eventSource.fire("Event after 10 seconds")
	}

	val eventStream: EventStream[String] = eventSource
\end{lstlisting}

Taip pat \textit{EventSource} (įvykių šaltinis) turi naudingą poklasį \textit{Timer} (laikrodis). Jo realizacija pavaizduota \ref{timer} kodo pavyzdyje. Šis sukuria tiksėjimo įvykius duotu laiko intervalu.

\begin{lstlisting}[caption=- įvykių srauto sukūrimas, label=timer]
	val timer = new Timer(0, 2000, {t =>  t >= 32000})

	for(t <- timer)
    	yield "timer: " + t.toString
\end{lstlisting}

Įvykių srautas valdo kolekciją klausytojų, tačiau konceptualiai reikėtų mąstyti kitaip. Klausytojo pridėjimas iš tikrųjų reiškia funkcijos iškvietimą kiekvienam kilusiui įvykiui, kitais žodžiais kiekvienam įvykių srauto įvykiui. Lygiai taip pat kaip Scala programavimo kalboje norint įvykdyti funkciją kiekvienai kolekcijos reikšmei yra iškviečiama \textit{foreach} funkcija. Tačiau įvykių srauto atveju, \textit{foreach} grąžina rezultatą akimirksniu, o funkcija yra išsaugoma ir vėliau įvykdoma kai tik iššaunamas koks nors įvykis. Šio atvejo praktinis variantas pademonstruotas \ref{foreach} kodo pavyzdyje.

\begin{lstlisting}[caption=- klausytojų pridėjimas, label=foreach]
	val eventSource = new EventSource[String] {}
	  
	// The following is syntactic sugar for
	// eventSource.foreach(event => alert("You fired: '" + event + "'"))
	for(event <- eventSource) {
	 	alert("You fired: '" + event + "'")
	}

\end{lstlisting}

Rinkinys transformacijos metodų įvykių srautą padaro labai universaliu. Šios transformacijos grąžina modifikuotą įvykių srautą. Transformacijas galima atlikti viena po kitos. Tai primena transformacijas galimas su Scala kolekcijomis, pavyzdžiui \ref{collectionTransformation} kodo pavyzdyje. Kai tik originalus įvykių srautas gauna įvykį, transformuotas įvykių srautas iššauna savo įvykį pritaikant klausytojo funkciją.

\begin{lstlisting}[caption=- kolekcijos transformacijos, label=collectionTransformation]
	List(1,2,3).map(_ * 10).filter(_ < 25)
\end{lstlisting}

Jeigu įvykių srautas gauna labai daug įvykių, tačiau aktualūs yra tik dalis jų, galima naudoti filtravimą, tai yra \textit{filter} metodą. Tai demonstruojama \ref{filter} kodo pavyzdyje. Čia kiekvienam įvykiui yra pritaikomas predikatas. Jeigu jis įvertinimas kaip teisingas arba kitaip \textit{true}, įvykis yra iššaunamas transformuotame įvykių sraute.

\begin{lstlisting}[caption=- įvykių srauto filtravimas, label=filter]
	val eventSource = new EventSource[String] {}
	eventSource.filter(_.length < 5)
\end{lstlisting}

Kitas esminis kolekcijų metodas yra \textit{map}. Jis leidžia transformuoti kolekciją pritaikant tam tikrą funkciją kiekvienam kolekcijos elementui. Pritaikius yra grąžinama nauja modifikuota kolekcija. Lygiai taip pat šis metodas gali būti pritaikytas įvykių srautui. Šis atvejis demonstruojamas \ref{map} kodo pavyzdyje. 

\begin{lstlisting}[caption=- įvykių visiškas transformavimas, label=map]
	val eventSource = new EventSource[String] {}
	eventSource.map(_.reverse)
\end{lstlisting}


Įvykių srautam taip pat galima pritaikyti \textit{flatmap} metodą. Scala kolekcijose šis metodas pritaiko funkciją visiems elementams, kuri grąžina naują kolekciją, kurioje sujungiamos visos grąžintų kolekcijų reikšmės. Pavyzdys pateiktas kodo pavyzdyje \ref{collectionsFlatmap}. Operacijos \textit{map} atveju būtų grąžinta reikšmė \textit{List(List(10, 11, 12), List(20, 21, 22), List(30, 31, 32))}, tačiau operacija \textit{flatMap} sujungia grąžintų kolekcijų reikšmes. 

\begin{lstlisting}[caption=- flatmap Scala kolekcijose, label=collectionsFlatmap]
	val original = List(1, 2, 3)
	val flatMapped = original.flatMap(x => List(x*10,x*10+1,x*10_2))
	flatMapped == List(10,11,12,  20,21,22,  30,31,32) 
\end{lstlisting}

Panašiai veikia \textit{flatMap} metodas įvykių srautuose. Pastarasis leidžia sukurti įvykių srautą, kuris iššauna įvykius, kurie yra iššauti kitų įvykių srautų. Metodas demonstruojamas \ref{flatmap} kodo pavyzdyje. Čia figūra yra išplečiama, o po to išnyksta.

\begin{lstlisting}[caption=- flatmap įvykių sraute, label=flatmap]
	// Assuming Shape is a case class with scale and opacity values
	// and millisTimer fires events once per millisecond, starting
	// at zero.
	// Scale should animate from 0 to 1 over the first second,
	// and opacity should animate from 1 to 0 over the next.
	def compositeAnimation(millisTimer: EventStream[Long], shape: Shape): EventStream[Shape] = {
	  val scale: EventStream[Double] =
	    millisTimer.map(m => m/1000.0)
	  val opacity: EventStream[Double] =
	    millisTimer.map(m => 1 - (m-1000)/1000.0)
	  val seconds = millisTimer.filter(_ % 1000 == 0).map(_ / 1000).
	    takeWhile(_ < 2000)
	  
	  seconds.flatMap {
	    case 0 => scale
	    case 1 => opacity
	  }
	}
\end{lstlisting}

Imperatyviose kalbose, dažna užduotis yra iteruoti per masyvą ir atlikti tam tikrus veiksmus. Šiam veiksmui atlikti naudojamas tam tikras skaičius kintamųjų veiksmam atlikti, pavyzdžiui skaičiavimam. Funkciniame programavime šiam tikslui dažnai yra naudojamas \textit{foldLeft} metodas. Pastarasis priima pradinę reikšmę arba kitaip būseną ir funkciją, kuri priima paskutinę reikšmę arba būseną ir kitą kolekcijos elementą. Kiekvienam elementui funkcija grąžina reikšmę arba būseną, kuri naudojama sekančiam funkcijos kvietimui. Verta pastebėti, jog funkcinis programavimas čia leidžia kodą vykdyti lygiagrečiai jo visiškai nekeičiant. \ref{collectionsFoldLeft} kodo pavyzdys demonstruoja sąrašo elementų sumos skaičiavimą.

\begin{lstlisting}[caption=- sąrašo elementų skaičiavimas naudojant flatmap, label=collectionsFoldLeft]
	list.foldLeft(0){(totalSoFar, nextElement) => totalSoFar + nextElement}
	//more commonly written as list.foldLeft(0)(_ + _)
\end{lstlisting}

Panašiai \textit{foldleft} metodą galima pritaikyti įvykių srautam. Demonstracija \ref{foldleft} kodo pavyzdyje.

\begin{lstlisting}[caption=- foldleft metodas įvykių sraute, label=foldleft]
	case class AvgState(total: Double, count: Int)
	val eventSource = new EventSource[String] {}
	
	eventSource.foldLeft(AvgState(0,0)){
      case (AvgState(total, count), s) => AvgState(total+s.length,count+1)
    } map {
      case AvgState(total, count) =>
        "Average length so far: " + (total/count)
    }
\end{lstlisting}

Įvykių srautus galima ne tik transformuoti, bet ir komponuoti, tai yra sujungti. Demonstracija \ref{merge} kodo pavyzdyje. Įvykių srautas \textit{allClicks} iššauna įvykį kai bet kuris iš kitų įvykių srautų iššauna įvykį.

\begin{lstlisting}[caption=- įvykių srautų sujungimas, label=merge]
	val allClicks = leftClicks | middleClicks | rightClicks
\end{lstlisting}

Įvykių srautas atvaizduoja įvykius kaip srautą diskrečių reikšmių laiko tekmėje, taigi šios reikšmės egzistuoja momentiškai. Tačiau įvykių srautą taip pat galima paversti signalu. Signalas atvaizduoja besitęsiančią reikšmę, tai yra turi dabartinę reikšmę. Kai signalo reikšmė pasikeičia, įvykių srautas iššauna naują įvykį. Įvykių srautą norint paversti signalu galima naudoti \textit{hold} metodą, kuriam yra paduodama pradinė reikšmė. Demonstracija \ref{hold} kodo pavyzdyje.

\begin{lstlisting}[caption=- įvykių srautų pavertimas signalu, label=hold]
	val eventSource = new EventSource[String] {}
  	val signal = eventSource.hold("(initial value of signal)")
  
  	def render = Demos.eventSourceInput(eventSource) ++
    	<p>Signal value: '{Demos.signalOutput(signal)}'</p>
\end{lstlisting}

\subsection{Įvykių kaupimas}

Šiame skyriuje yra aprašomos žinios apie įvykių kaupimą, pliusus ir minusus, įvykių srautus bei įvykių kaupimą funkciniame programavime remiantis daugiausia Vaughn Vernon surinkta ir aprašyta informacija \cite{vernon2013implementing}.

\subsubsection{Įvadas}

Kartais verslui svarbu fiksuoti objekto pasikeitimus domeno modelyje\footnote{http://en.wikipedia.org/wiki/Domain\_model}. Šiuos pasikeitimus galima stebėti skirtingais būdais. Įprastai yra pasirenkama stebėti kai esybė\footnote{http://en.wikipedia.org/wiki/Entity} yra:

\begin{itemize}

	\item sukurta,

	\item paskutinį kartą modifikuota

	\item bei kas atliko modifikaciją.

\end{itemize}

Tačiau šis būdas nepateikia jokios informacijos apie vienkartinius pasikeitimus.

Atsiradus poreikiui stebėti pasikeitimus detaliau, verslas reikalauja dar daugiau metaduomenenų\footnote{http://en.wikipedia.org/wiki/Metadata}, ko pasekoje tokie faktai kaip individualios operacijos laiko tekmėje bei jų įvykdymo laikas tampa svarbūs. Šie poreikiai verčia įvesti audito žurnalą fiksuoti labai tikslias panaudojimo atvejų metrikas, tačiau pastarasis būdas turi apribojimų. Jis gali atskleisti dalį informacijos apie tai kas nutiko sistemoje, leisti rasti bei ištaisyti dalį riktų bei klaidų programinėje įrangoje. Bet audito žurnalas neleidžia patikrinti domeno objekto būsenos prieš ir po tam tikrų pasikeitimų. O jeigu būtų galima išgauti daugiau informacijos iš pasikeitimų stebėjimo?

Visi programinės įrangos kūrėjai susiduria su labai tiksliu pasikeitimų stebėjimu. Įprastas ir populiarus pavyzdys yra išeities kodo saugyklos, tokios kaip CVS\footnote{http://www.nongnu.org/cvs/}, Subversion\footnote{http://subversion.apache.org/}, Git\footnote{http://git-scm.com/} arba Mercurial\footnote{http://mercurial.selenic.com/}. Visos šios pataisų valdymo sistemos leidžia stebėti pirminių failų pasikeitimus. Įrankiai leidžia peržiūrėti išeities kodo artefaktus nuo pačios pirmosios pataisos iki paskutinės. Kai visi išeities failai yra nusiunčiami į pataisų kontrolės sistemą, ši gali stebėti pasikeitimus viso programinės įrangos kūrimo gyvavimo ciklo metu.

Jeigu šis principas būtų pritaikytas vienai esybei, tada vienam agregatui\footnote{http://martinfowler.com/bliki/DDD\_Aggregate.html} bei galiausiai kiekvienam modelio agregatui, galima suprasti kokią naudą atneša sistemos objektų pasikeitimų stebėjimas:

\begin{itemize}

	\item Kas būtent nutiko modelyje, jog agregato egzempliorius buvo sukurtas?

	\item Kas nutiko agregato egzemplioriui bėgant laikui? (Operacijų požiūriu)

\end{itemize}

Turint visų atliktų operacijų istoriją, galima palaikyti laikinus modelius. Toks kaitos stebėjimas yra įvykių kaupimo principas. \ref{pic:es} diagramoje pateika šio principo aukšto lygio reprezentacija. Agregatai publikuoja įvykius, kurie yra išsaugomi įvykių saugykloje ir naudojami sekti modelio būsenos pasikeitimus. Verta paminėti, jog įvykiai reprezentuoja tam tikrą būsenos pasikeitimą bėgant laikui, todėl jie nėra atnaujinami arba ištrinami. Saugykla nuskaito įvykius iš įvykių saugyklos ir pritaiko juos vieną po kito taip atkurdama agregato būseną. 

\begin{figure}[ht]
	\centering
	\includegraphics[width=0.9\linewidth]{pics/es.png}
	\caption{Įvykių kaupimo aukšo lygio reprezentacija}
	\label{pic:es}
\end{figure}

\subsubsection{Momentinė kopija}

Ilgame periode sistemoje susikaupia daugybė įvykių. Atkuriant agregato būseną reikia atkartoti šimtus, tūkstančius ar net milijonus įvykių. Tai tampa šio modelio silpnąja puse, nes įvykių atkartojimas užtrunka vis ilgiau sistemai plečiantis.

Tačiau šio duomenų kamsčio galima išvengti naudojant agregato būsenos momentines kopijas. Tam tikrame įvykių saugyklos istorijos taške yra padaroma agregato būsenos kopija. Serializuota agregato būsena yra įrašoma į įvykių saugyklą. Nuo to momento, agregatas yra atkuriamas pirmiausia naudojantis naujausia jo būsenos momentine kopija ir tik po to atkartojami visi naujesni įvykiai.

Momentinės kopijos nėra atkuriamos atsitiktinai. Jos gali būti kuriamos kas apibrėžtą skaičių įvykių. Šis skaičius turėtų būti parinktas analizuojant domeno sritį bei sistemą ir radus optimalų variantą. Tikėtinai gali būti 50 arba 100 įvykių tarp momentinių kopijų.

\subsubsection{Įvykių kaupimo privalumai ir trūkumai}

Kaip saugojimo mechanizmas, įvykių kaupimas stipriai skiriasi ir pakeičia ORM\footnote{http://www.orm.net/} įrankį. Kadangi įvykiai dažnai įrašomi kaip dvejetainės reprezentacijos, jie negali būti optimaliai naudojami užklausoms atlikti. Faktiškai įvykių kaupimu pagrįstoms saugykloms tereikia vienos operacijos - gauti įrašus pagal unikalią agregato tapatybę \cite{CQRS:GregYoung}. To pasekoje užklausom daryti reikia kito kelio. Dažniausiai tam pasirenkamas CQRS\footnote{http://martinfowler.com/bliki/CQRS.html} principas \cite{Betts:2013:ECE:2509680}. 

Įvykių kaupimas verčia kitaip mąstyti apie domeno modelį. Įvykių istorija gali padėti surasti bei ištaisyti sistemos defektus bei klaidas \cite{SeanFitz2012}. Derinimas naudojant istoriją visų veiksmų, kurie nutiko sistemoje, turi didžiulį pranašumą. Įvykių kaupimas gali vesti prie didelio našumo domeno modelių, tai yra palaikyti ypač didelį skaičių operacijų per sekundę. Pavyzdžiui, įrašymas į vieną duomenų saugyklos lentelę yra ypač greitas. Negana to, tai leidžia CQRS užklausų modelį išplėsti horizontaliai, nes duomenų šaltinio atnaujinimai įvykdomi fone, kai įvykių saugykla yra atnaujinama naujais įvykiais.

\subsubsection{Įvykių kaupimas funkciniame programavime}

Vaughn Vernon pateikia keletą pastebėjimų apie įvykių kaupimą funkciniame programavime, kurie gali būti naudingi atliekant projektinius sprendimus bei eksperimentinį tyrimą:

\begin{itemize}

	\item Agregatas projektuojamas kaip nekintantis būsenos įrašas kartu su funkcijomis, kurios keičia būseną. Šios funkcijos paprasčiausiai priima būsenos įrašą ir įvykių argumentus ir gražina naują būsenos įrašą kaip rezultatą. Tokia funkcija pavaizduota \ref{aggregate} kodo pavyzdyje.

\begin{lstlisting}[caption=- agregato būsenos keitimas, label=aggregate]

	Funkcija<Busena, Ivykis, Busena>

\end{lstlisting}

	\item Dabartinė agregato būsena gali būti apibrėžta kaip suskleidimas į kairę visų praeities įvykių, kurie yra perduodami būseną keičiančiai funkcijai.

	\item Agregato metodai gali būti išreikšti kaip kolekcija funkcijų be būsenos.

	\item Įvykių saugykla gali būti suvokiama bei naudojama kaip funkcinė duomenų bazė, nes ji perduoda argumentus funkcijoms, kurios keičia agregato būseną. Momentinės kopijos įvykių saugykloje primena įsiminimą atmintyje\footnote{http://en.wikipedia.org/wiki/Memoization} funkciniame programavime.

\end{itemize}

\subsection{Monados}

\subsubsection{Terminas}

Funkciniame programavime monada yra struktūra, kuri atspindi skaičiavimus, apibrėžtus kaip seka žingsnių. Tipas kartu su monados struktūra apibrėžia ką reiškia vykdyti operacijas viena po kitos bei naudoti to pačio tipo įdėtines funkcijas. Tai leidžia kurti komandų grandines, kurios pažingsniui apdoroja informaciją, kur kiekvienas veiksmas yra dekoruojamas naujomis apdorojimo taisyklėmis, kurias apibrėžia monada \cite{OSullivan:2008:RWH:1523280}. Taip pat monada gali būti laikoma kaip funkcinis projektavimo šablonas, skirtas kurti daugybinius tipus \cite{monadicDesign}.

\subsubsection{Taisyklės}

Anot \cite{Wadler:1995:MFP:647698.734146}, monadų operacijos turi tenkinti tris taisykles:

\begin{itemize}

	\item Kairiojo vienetotapatybės (angl. left unit/identity) - suskaičiuojama reikšmė \textit{a}, rezultatui priskiriamas \textit{b} ir suskaičiujama \textit{n} reikšmė. Rezultatas yra toks pat kaip \textit{n} kai \textit{a} reikšmė pakeičiama \textit{b} reikšme. Parodyta \ref{pic:left} paveikslėlio formulėje.

\begin{figure}[ht]
	\centering
	\includegraphics{pics/left.png}
	\caption{Perkėlimo į kairę taisyklė}
	\label{pic:left}
\end{figure}

	\item dešiniojo vieneto/tapatybės (angl. right unit/identity) - suskaičiuojama reikšmė \textit{m}, rezultatas priskiriamas \textit{a} ir grąžinama \textit{a}. Rezultatas yra toks pat kaip \textit{m}. Parodyta \ref{pic:right} paveikslėlio formulėje.

\begin{figure}[ht]
	\centering
	\includegraphics{pics/right.png}
	\caption{Perkėlimo į dešinę taisyklė}
	\label{pic:right}
\end{figure}

	\item Asociatyvumas - suskaičiuojama reikšmė \textit{m}, rezultatas priskiriamas \textit{a}, suskaičiuojamas \textit{n}, rezultatas priskiriamas \textit{b}, suskaičiuojama \textit{o}. Skliaustelių padėtis šiuo atveju yra nesvarbi. Kintamojo \textit{a} apimtis įtraukia \textit{o} kairėje, tačiau neįtraukia \textit{o} dešinėje, todėl ši taisyklė galioja tik tada kai \textit{a} nėra laisvas nuo \textit{o}. Parodyta \ref{pic:associative} paveikslėlio formulėje.

\begin{figure}[ht]
	\centering
	\includegraphics{pics/associative.png}
	\caption{Asociatyvumas}
	\label{pic:associative}
\end{figure}

\end{itemize}

\subsubsection{Tolesnis darbas}

Kitame mokslinio tyrimo etape ruošiamąsi įrodyti, kad funkcinį-reaktyvų programavimą įmanoma taikyti įvykių kaupimo sistemose. Šiam tikslui pasiekti toliau bus nagrinėjamos monados, jų komponavimo būdai. Kad geriau suprasti monadas turbūt prireiks įgauti žinių kategorijų teorijoje, daugiau žinių apie funkcinį programavimą, pavyzdžiui tipizuotos klasės, parametrizuoti tipai ir t.t.

\subsection{Išvados}

Literatūros analizės metu remiantis kitų autorių patirtimi:

\begin{itemize}

\item išnagrinėtas funkcinis-reaktyvus programavimas,

\item išnagrinėtas įvykių kaupimo principas,

\item išnagrinėti įvykių srautai bei operacijos su jais,

\item susipažinta su įvykių kaupimu funkciniame programavime,

\item susipažinta su monadomis,

\item įvaldyta sąvokų sistema, susijusi su nagrinėjama tematika.

\end{itemize}


% \section{NAUJA!!!!!}
% \input{sections/new_info}

\subsection{Funkcinis-reaktyvus programavimas}

\subsubsection{Įvadas}

Pagrindinė funkcinio programavimo abstrakcija yra funkcija, kuri priima tam tikrą įvestį, o jos rezultatas yra tam tikra išvestis. Funkcijos įvestis bei rezultatas gali būti kita funkcija. Funkcinėje programavimo kalboje funkcijos yra pirmos eilės reikšmės.

Priešingai, labiau populiarus imperatyvaus programavimo modelis priima sakinius arba veiksmus kaip pirminius programos konstravimo blokus, kurie modifikuoja būseną. Tokios programos turi nuoseklų kontrolės srautą ir reikalauja samprotavimo apie pašalinius efektus. Iš prigimties jie yra atsparūs kompoziciniam samprotavimui.

Netgi funkcinėse programavimo kalbose, reaktyvios programos yra paprastai parašytos imperatyviu stiliumi, naudojant žemo lygio bei ne paruoštų komponentų abstrakcijas įskaitant atgalinius skambintojus arba objektu paremtais įvykių dorokliais. Tai pririša interaktyvumo modelį prie prie žemo lygio realizacijos detalių tokių kaip laikas bei įvykių apdorojimo modeliai.

Anot \cite{Survey}, funkcinis reaktyvus programavimas, toliau vadinamas tiesiog FRP reiškia, jog modelis turi išlaikyti funkcinio programavimo charakteristikas (pavyzdžiui, primityvios kalbos konstrukcijos turi likti pirmosios klasės) įtraukiant reaktyvumą į kalbos modelį. Funkcinis-reaktyvus programavimas yra būdas modeliuoti reaktyvų (besikeičiantis laiko tekmėje bei reaguojantis į išorinį stimulą) elgesį visiškai funkcinėse programavimo kalbose. FRP leidžia deklaratyviu ir paprastu būdu modeliuoti sistemas, kurios turi reaguoti į duomenis bėgant laikui.

\subsubsection{Pagrindinis tikslas}

Pagrindinis funkcinio-reaktyvaus programavimo tikslas \cite{Survey}:

\begin{itemize}

	\item saugus programavimas - kompiliatorius turi kiek įmanoma patikrinti programų korektiškumą;

	\item efektyvus programavimas - programos turėtų veikti realiu laiku, todėl efektyvios ir optimizuotos operacijos yra būtinos;

	\item komponavimas - suteikti kompozicines ir aukšto lygio abstrakcijas, jog būtų galima kurti reaktyvias programas. FRP leidžia kurti programas iš smulkesnių programų, o ne orientuotą į problemą, vientisą kodą.

\end{itemize}

\subsubsection{Sąvokos}

Pagrindinės FRP sąvokos yra \cite{ElliottHudak97:Fran}:

\begin{itemize}

	\item signalai arba elgsena - besikeičiančios laike reikšmės;

	\item įvykiai - kolekcija momentinių reikšmių arba laiko-reikšmės poros.

\end{itemize}

FRP pasiekia reaktyvumą naudodamas konstrukcijas, kurios tiksliai apibrėžia kaip signalai arba elgsena pasikeičia reaguodami į įvykius. Tai yra pagrindinis būdas išreiškiant bei realizuojant elgseną. Kitu būdu, elgsena gali būti laiko semantinės funkcijos, kuriose laikas yra pakeičiamas kilus įvykiui \cite{Nilsson:2002:FRP:581690.581695}.

%%%%%%%%START SEMANTIKA

\subsubsection{Formali semantika}

Šiame skyriuje bus apibrėžta formali elgesio ir įvykių domeno sritys bei jų kombinatoriai \cite{ElliottHudak97:Fran}.

\paragraph{Semantinės domeno sritys}

Semantinė domeno sritis - kalbos konstruktų formalizavimas į matematinius objektus. Tai suteikia konstruktams prasmę bei įgalina samprotauti. Mus dominanti abstrakti laiko domeno sritis yra vadinama \textit{Time}. Abstrakti polimorfinio elgesio(\textit{$\alpha$-behaviors}) domeno sritis yra žymima \textit{Behavior_{\(\alpha\)}}, o polimorfinių įvykių(\textit{$\alpha$-events}) yra žymima \textit{Event_{\(\alpha\)}}.

Daugiausia šių sričių (sveikieji skaičiai, loginės reikšmės) yra standartinės ir nereikalauja paaiškinimo. \textit{Time} domeno sritis reikalauja specialaus traktavimo kadangi laiko reikšmės įtraukia dalinius elementus (angl. partial element). Ypač yra žinoma, jog laikas bent jau kažkokia reikšmė netgi nežinant galutinės reikšmės. Tiksliau laiko domeno sritį galima apibrėžti taip: tarkime \textit{R} yra rinkinys realių skaičių ir jame yra elementai \(\infty\)  ir -\(\infty\). Šis rinkinys turi standartinį aritmenį rikiavimą \(\leq\) įskaitant faktą, jog -\(\infty\) \(\leq\) \(\infty\) kiekvienam x \(\in\) \textit{R}.

Dabar apibrėžkime laiką kaip \textit{Time} = \texit{R} + \textit{R}, kur elementai antrame rinkinyje \textit{R} yra atskiriami pridėjus priešdėlį \(\geq\) (pavyzdžiui, \(\geq\) 42 skaitytume kaip ,,daugiau arba lygu 4''". Tada galima apibrėžti \perp_{Time} =  \(\geq\)(-\(\infty\)) ir domeno srities (pavyzdžiui, informacijos) rikiavimą pagal laiką\footnote{A \(\sqsubseteq\) B reiškia, jog kiekvienas aibės A elementas yra ir aibės B elementas}:

\begin{gather*}
x \(\sqsubseteq\) x, \(\forall\)x \in R \\
\(\geq\)x \(\sqsubseteq\) y  if x \(\leq\) y, \(\forall\)x,y \in R \\
\(\geq\)x \(\sqsubseteq\) \(\geq\) y if x \(\leq\) y, \(\forall\)x,y \in R
\end{gather*} 

Lengva pastebėti, jog $\perp_{Time}$ yra tikrai apatinis elementas. Svarbu paminėti, jog \textit{y} yra tik bent viršutinė dalinių elementų rinkinio riba (angl. least upper bound), kuri apytiksliai lygi:

\begin{gather*}
y = \sqcup\ \{ \geq\ \(\|\) x \leq\ y \}
\end{gather*} 

Kadangi laiko domeno rikiavimas yra grandinės tipo ir kiekviena tokia grandinė turi bent viršutinę ribą (prisiminkime \textit{R} turi viršutinį elementą \(\infty\)), laiko domenas yra pilnos dalinės tvarkos (angl. complete partial order). Šis faktas yra būtinas, jog būtų galima užtikrinti, kad rekursyvūs apibrėžimai yra gerai apibrėžti.

\textit{Time} rinkinio elementai naudingiausi įvertinant laiką kada atsitinka įvykis. Įvykis kurio laikas apytiksliai \(\geq\)t yra tas, kurio konkretus įvykimo laikas yra didesnis nei \textit{t}. Svarbu, jog įvykio, kuris niekada neįvyksta, laikas yra \(\infty\), bent viršutinė \textit{R} riba.

Galiausiai apibrėžimą galima išplėsti aritmetiniui operatoriui \(\leq\) visam \textit{Time} apibrėžiant jo elgesį visuose subdomenuose:

\begin{gather*}
x \(\leq\) _\(\geq\)y if x \leq y
\end{gather*}

Tai gali būti skaitoma kaip: ,,Laikas x yra mažesnis arba lygus laikui, kuris yra bent y jeigu x mažiau arba lygu y''. Lengva parodyti, jog ši išplėsta tipo \textit{Time} \(\rightarrow\) \textit{Time} \(\rightarrow\) \textit{Bool} funkcija yra tolydi atsižvelgiant į \(\sqsubseteq\).

\paragraph{Semantinės funkcijos}

Polimornio elgesio interpretaciją galima apibrėžti kaip funkciją, kuri priima polimorfinę reikšmę bei pagamina elgesio b reikšmę  laiku t.

\begin{gather*}
at : Behavior_{\alpha} \(\rightarrow\) Time \(\rightarrow\) \alpha
\end{gather*}

Tada galima apibrėžti polimorfinių įvykių interpretaciją kaip paprastą ir negriežtą \textit{Time} \(\times\) \(\alpha\) porą, aprašančią laiką ir informaciją, susijusią su įvykio atsitikimu:

\begin{gather*}
occ : Event_{\alpha} \(\rightarrow\) Time \(\times\) \alpha
\end{gather*}

Žinant semantinę domeno sritį, galima pateikti formalias elgesio bei įvykių kombinatorių interpretacijas.

\paragraph{Elgesio semantika}

Elgesys yra kuriamas iš kito elgesio, statinių (nesikeičiančių laike) reikšmių ir įvykių naudojantis kolekcija konstruktorių (kombinatorių).

\begin{itemize}
	
	\item \textbf{Laikas}. Paprasčiausias primityvus elgesys yra laikas - \texit{time}, kurio semantika yra:

\begin{gather*}
time: Behavior_{Time} \\
\textbf{at}[[time]]t = t
\end{gather*}

	Šiuo atveju \textbf{at}[[time]]t = t yra tik \textit{Time} tapatumo funkcija.

	\item \textbf{Pakėlimas\footnote{Praktikoje pakėlimas yra reikalingas gana dažnai, todėl būtų nepatogu visur kelti išreikštinai. Vietoje to pageidautina naudoti žinomus vardus, tokius kaip: ,,sin'', ,,cos'', ,,+'', ,,*''\ ir netgi literalus, tokius kaip ,,3''\ arba ,,mėlynas''\ nurodant pakeltas versijas. Pavyzdžiui, literalas ,,42''\ turėtų elgtis kaip nekintantis elgesys ,,$lift_{0}$\ 42'', o sudėtis ,,$b_{1}$ + $b_{2}$''\ kaip ,,$lift_{2}$ (+)\ $b_{1}$ $b_{2}$''}} (angl. lifting) - įprastas būdas funkcijoms, apibrėžiančioms nekintančias reikšmes, ,,pakelti'' į analogines funkcijas, apibrėžtoms elgesiu. Pakėlimas yra įvykdomas naudojant (konceptualiai begalinę) šeimą operatorių - vieną keikvienam funkcijos valentingumui (angl. arity).

\begin{gather*}
lift_{n} : ( \alpha_{1} \rightarrow\ ... \rightarrow\ \alpha_{n} \rightarrow\ \beta ) \rightarrow\
	Behavior_{\alpha_{1}} \rightarrow\ ... \rightarrow\ Behavior_{\alpha_{n}} \rightarrow\ Behavior_{\beta} \\
\textbf{at}[[lift_{n}\ f\ b_{1}...b_{n}]]t = f (\textbf{at}[[b_{1}t]])...(\textbf{at}[[b_{n}]]t)
\end{gather*}

	Svarbu paminėti, jog nesikeičiančios reikšmės pakėlimas yra \textit{lift_{0}}

	\item \textbf{Laiko transformacija}. Laiko transformacija leidžia vartotojui pakeisti lokalų laikotarpį. Toks būdas palaiko bet kokio elgesio laikiną modalumą. (Panašiai 2D ir 3D transformacijos palaiko erdvinį modalumą geometrijoje)

\begin{gather*}
timeTransform : Behavior_{\alpha} \rightarrow\ Behavior_{Time} \rightarrow\ Behavior_{\alpha} \\
\textbf{at}[[timeTransform\ b\ tb]] = \textbf{at}[[b]]\ o\ \textbf{at}[[tb]]
\end{gather*}

Taigi idėja yra, jog laikas yra \textit{timeTransform} tapatybė:

\begin{gather*}
timeTransform\ b\ time = b
\end{gather*}

Pavyzdžiui, laiko transformacijos pavyzdys Fran:

\begin{gather*}
timeTransform\ b\ (time/2)
\end{gather*}

sulėtina animaciją dvigubai.

	\item \textbf{Integracija}. Integracija pritaikoma tiek realių skaičių reikšmes turinčiam, tiek 2D ir 3D vektoriaus reikšmes turinčiam elgesiui, arba apskritai vektoriaus erdvei (limituota). Naudojant Haskell notaciją vektoriaus erdvei:

\begin{gather*}
integral: VectorSpace\ \(\alpha\) \Rightarrow\ Behavior_{\alpha} \rightarrow\ Time \rightarrow\ Behavior_{\alpha} \\ 
\textbf{at}[[integral\ b\ t_{0}]]t = [ \int_\(t_{0}\)^t\ \textbf{at}[[b]]]
\end{gather*}

	Integracija leidžia specifikuoti elgesio greitį, o naudojama du kartus pagreitį. Pavyzdžiui, judančio kamuoliuko greitis yra duotas kaip elgesys b (gali būti ir pastovus ir nepastovus), tada jo reliatyvi pozicija laiko pradžios atžvilgiu \(t_{0}\)\ yra duota kaip \textit{integral b \(t_{0}\)}. Tai leidžia natūraliai išreikšti fizika paremtas animacijas.

	\item \textbf{Reaktyvumas}. Pagrindinė sąveika yra tarp elgesio ir įvykių ir tai padaro elgesį reaktyviu. Konkrečiai elgesys \textit{b untilB e} parodo b elgesį iki tol kol įvyksta e ir tada pasikeičia į elgesį asocijuotą su įvykiu e. Formaliai:

\begin{gather*}
untilB : Behavior_{\alpha} \rightarrow\ Event_{Behavior_{\(\alpha)}} \rightarrow\ Behavior_{\alpha} \\
\textbf{at}[[b\ untilB\ e]]t = if\ t \leq t_{e} \textbf{ then at}[[b]]t\ \textbf{else at}[[b']]t \\
\textbf{where} (t_{e}, b') = \textbf{occ}[[e]]
\end{gather*}

\end{itemize}

\paragraph{Įvykių semantika}

\begin{itemize}

	\item \textbf{Įvykių apdorojimas}. Norint duoti pavyzdį naudojant specialią rūšį įvykių, pirmiausia reikia apibrėžti įvykių apdorotojus, kurie yra pritaikomi laikui ir su įvykiu susietiem duomenim naudojant šį operatorių:

\begin{gather*}
(+$\Rightarrow$) : Event_{\alpha} = (Time \rightarrow \alpha \rightarrow \beta) \rightarrow Event_{\beta}\\
\textbf{occ}[[e +$\Rightarrow$ f]] = (t_{e}, f\ t_{e}\ x)\\
\textbf{where}\ (t_{e},x) = \textbf{occ}[[e]]
\end{gather*}

	Patogumui galima naudoti šias išvestas operacijas, kurios ignoruoja laiką arba informaciją arba abu kartu:

\begin{gather*}
(=$\Rightarrow$) : Event_{\alpha} \rightarrow (\alpha \rightarrow \beta) \rightarrow Event_{\beta}\\
(*$\Rightarrow$) : Event_{\alpha} \rightarrow (Time \rightarrow \beta) \rightarrow Event_{\beta}\\
(-$\Rightarrow$) : Event_{\alpha} \rightarrow \beta \rightarrow Event_{\beta}\\
ev =\Rightarrow g = ev\ +\Rightarrow \lambda t\ x.\ g\ x\\
ev *\Rightarrow h = ev\ +\Rightarrow \lambda t\ x.\ h\ t\\
ev -\Rightarrow x' = ev\ +\Rightarrow \lambda x.\ x'
\end{gather*}

	(+$\Rightarrow$) gauna visus parametrus, (-$\Rightarrow$)  negauna parametrų, (*$\Rightarrow$) gauna tik laiką, o ($\Longrightarrow$) gauna tik informaciją.

	\item \textbf{Nekintantys įvykiai}. Paprasčiausia įvykių rūšis yra tiesiogiai specifikuoti pagal laiką ir reikšmę.

\begin{gather*}
constEv : Time \rightarrow \alpha \rightarrow Event_{\alpha}\\
\textbf{occ}[[constEv\ t_{e}\ x]] = (t_{e},x)
\end{gather*}

	Nors, pavyzdžiui, elgesys:

\begin{gather*}
b_{1}\ untilB\ (constEv\ 10\ x) -$\Rightarrow$ b_{2}
\end{gather*}

	parodo elgesį \(b_{1}\) iki laiko taško 10, kuriame jis pradeda rodyti elgesį \(b_{2}\) (x yra ignoruojamas šiame pavyzdyje, bet iš tikro neturėtų).

	\item \textbf{Išoriniai įvykiai}. Išoriniai įvykiai yra pavyzdžiui pelės mygtuko paspaudimas, kuris gali būti kairysis arba dešinysis mygtukas. Reikšmė, asocijuota su kurio nors mygtuko paspaudimu yra interpretuojama kaip atleidimo įvykis, kuris grąžina vienetinę reikšmę (() yra vieneto tipas):

\begin{gather*}
lbp, rbp : Time \rightarrow Event_{Event_{()}}
\end{gather*}

	Įvykio \textit{lbp \(t_{0}\)} prasmė, pavyzdžiui, yra pora (\(t_{e}\), e), tokia kad \(t_{e}\) yra pirmojo kairio mygtuko paspaudimas po laiko \(t_{0}\) ir \textit{e} yra įvykis, reiškiantis pirmojo kairiojo mygtuko atleidimą po laiko \(t_{e}\). Iš to seka, jog:

\begin{gather*}
b_{1}\ untilB\ (lbp, t_{0}) =$\Rightarrow$ \lambda e.\\
b_{2}\ untilB\ e (-\Rightarrow) \\
b_{3}
\end{gather*}

	parodo elgesį \(b_{1}\) iki kairiojo mygtuko paspaudimo, kurio metu jis tampa \(b_{2}\) kol kairysis mygtukas yra atleistas, o tada tampa \(t_{3}\).

	\item \textbf{Predikatai}. Natūralus būdas specifikuoti tam tikrus įvykius kaip pirmąjį laiką kada loginės reikšmės elgesys tampa tiesa (\textit{true}) po duoto laiko:

\begin{gather*}
predicate : Behavior_{Bool} \rightarrow Time \rightarrow Event_{()}\\
\textbf{occ}[[predicate\ b\ t_{0}]] = (inf\ \{t > t_{0}\ |\ \textbf{at}[[b]]t\}, ())
\end{gather*}

	Predikato įvykio laikas yra begalinis ir sudarytas iš rinkinio laiko taškų didesnių tei \(t_{0}\), kuriuose elgesys yra teigiamas. Šis laikas gali būti ir \(t_{0}\).

	Tada elgesys:

\begin{gather*}
b_{1}\ untilB\ (predicate\ (sin\ time = 0.5)\ t_{0}) -$\Rightarrow$ b_{2}
\end{gather*}

 	parodo \(b_{1}\) iki pirmojo laiko taško t po \(t_{0}\), kur \textit{sin t} yra 0.5, po kurio jis parodo \(b_{2}\).

 	\item \textbf{Pasirinkimas}. Galima pasirinkti ankstesnįjį iš dviejų įvykių naudojantis operatoriumi .|.:

\begin{gather*}
(.|.) : Event_{\alpha} \rightarrow Event_{\alpha} \rightarrow Event_{\alpha}\\
\textbf{occ}[[e\ .|.\ e']] = (t_{e},x), \textbf{if}\ t_{e} \leq t'_{e}\\
				= (t'_{e}, x'),\ otherwise\\
\textbf{where}\ (t_{e},x) = \textbf{occ}[[e]]\\
\textbf{where}\ (t'_{e},x) = \textbf{occ}[[e']]
\end{gather*}

	Pavyzdžiui, elgesys:

\begin{gather*}
b_{1}\ untilB\ (lbp\ t_{0}\ .|.\ predicate\ (time > 5)\ t_{0}) -$\Rightarrow$\ b_{2}
\end{gather*}

	laukia kol arba bus nuspaustas kairysis mygukas arba praeis 5 sekundės prieš pakeičiant \(b_{1}\) elgesį į \(b_{2}\). Kaip alternatyva, sekantis pavyzdys pakeičia elgesį į \(b_{3}\) po skirto laiko pabaigos:

\begin{gather*}
b_{1}\ untilB\ (lbp\ t_{0}\ -$\Rightarrow$\ b_{2}\ .|.\ predicate\ (time > 5)\ t_{0}) -$\Rightarrow$\ b_{3}
\end{gather*}

	\item \textbf{Momentinė kopija}. Tuo momentu kai nutinka įvykis yra dažnai patogu padaryti elgesio reikšmės momentinę kopiją tam tikrame laiko taške. Tai formaliai galima užrašyti:

\begin{gather*}
snapshot : Event_{\alpha} \rightarrow Behavior_{\beta} \rightarrow Event_{\alpha \times \beta}\\
\textbf{occ}[[e\ snapshot\ b]] = (t_{e}, (x, \textbf{at}[[b]]t_{e}))\\
\textbf{where}\ (t_{e}, x) = \textbf{occ}[[e]]
\end{gather*}

	Pavyzdžiui, elgesys:

\begin{gather*}
b_{1}\ untilB\ (lbp\ t_{0}\ snapshot\ (sin\ time)) =$\Rightarrow$\ \lambda (e,y).\ b_{2}
\end{gather*}

	paima laiko, kai yra nuspaudžiamas kairysis mygtukas, sinusą, priskiria jį \textit{y} ir seka elgesiu \(b_{2}\), kuris galimai priklauso nuo \textit{y}. Nepaisant to, šį pavyzdį taip pat būtų galima realizuoti paimant kairiojo mygtuko paspaudimo įvykio laiką ir skaičiuojant sinusą, bendru atveju elgesio buvimas momentine kopija yra ganėtinai sudėtingas ir gali priklausyti nuo išorinių įvykių.

	\item \textbf{Įvykių sekos} Kartais yra naudinga naudoti vieną įvykį, kad būtų galima sukurti kitą. Įvykis \textit{joinEv e} yra įvykio, kuris nutinka kai nutinka \textit{e'}, kur \textit{e'} yra \textit{e} reikšmės dalis:

\begin{gather*}
joinEv: Event_{Event_{\alpha}} \rightarrow Event_{\alpha}\\
\textbf{occ}[[joinEv\ e]] = \textbf{occ}[[snd\ (\textbf{occ}[[e]])]]
\end{gather*}

	Pavyzdžiui, įvykis:

\begin{gather*}
joinEv\ (lbp\ t_{0}\ *\Rightarrow predicate\ (b = 0))
\end{gather*}

	nutinka pirmą kartą kai elgesys \textit{b} turi nulinę reikšmę po pirmojo kairiojo mygtuko paspaudimo po laiko tarpo \(t_{0}\).

\end{itemize}

\textbf{Panaudojimo pavyzdys}

Remiantis primityvių kombinatorių pavyzdžiais elgesiui bei įvykiam kartu su jų formalia semantika, galima pateikti konkretų pavyzdį Haskell programavimo kalboje. \ref{sign} pateiktame pavyzdyje yra skaitine reikšme išreikštas elgesys, kurio pradinė reikšmė yra 0 ir tampa -1 jeigu yra nuspaustas kairys pelės mygtukas (lbp - left button pressed) arba 1 jeigu yra nuspaustas dešinys pelės mygtukas (rbp - right button pressed).

\begin{lstlisting}[caption=- signalo funkcija nuo pelės mygtuko paspaudimo, label=sign]
	bSign t0 =
		0 'untilB' lbp t0 ==> nonZero (-1) .|.
				   rbp t0 ==> nonZero 1
		where nonZero r stop =
				r 'untilB' stop *=> bSign
\end{lstlisting}

Šį elgesį galima panaudoti paveikslėlio dydžio keitimui. Kairio arba dešinio pelės mygtuko paspaudimas priverčia paveikslėlį padidėti arba sumažėti. Pavyzdys pateiktas \ref{grow}.


\begin{lstlisting}[caption=- paveiksliuko dydžio modifikavimas pelės paspaudimu, label=grow]
	bSign t0 =
		0 'untilB' lbp t0 ==> nonZero (-1) .|.
				   rbp t0 ==> nonZero 1
		where nonZero r stop =
				r 'untilB' stop *=> bSign
\end{lstlisting}



%%%%%%%%%%%%END SEMANTIKA

\subsubsection{Sąvybės}

Anot anksčiausios funkcinio-reaktyvaus programavimo formuluotės \cite{ElliottHudak97:Fran}, pagrindinės sąvybės, kuriomis pasižymi FRP:

\begin{itemize}

	\item elgsenos arba signalų modeliavimas bėgant laikui,

	\item įvykių, kurie turi baigtinį skaičių atsitikimų daugelyje laiko taškų, modeliavimas,

	\item perjungimas (angl. switching) - sistema gali pasikeisti dėl atsitikusių įvykių,

	\item analizės detalių, tokių kaip reaktyvaus modelio įvykių ėmimo dažnis, atskyrimas.

\end{itemize}

%%%%%%%%%%%%%%%PRIVALUMAI IR ESME START


\subsubsection{Modeliavimo privalumai lyginant su žemo lygio prezentacijos detalėmis}

Anot \cite{ElliottHudak97:Fran} modeliavimo privalumai prieš prezentacijos detales (pavyzdžiui nuotraukos pozicijos valdymas cikle) yra panašūs į funkcinės (arba galima sakyti deklaratyvios) programavimo kalbos paradigmą ir apima aiškumą, kūrimo lengvumą, komponavimą ir švarią semantiką. Be šių yra programai būdingų privalumų, tam tikrais atvejais patrauklesnių iš programinės įrangos kūrėjo bei galutinio vartotojo perspektyvos. Šie privaluai apima:

\begin{itemize}

	\item Kūrimas - turinio kūrimo sistemos natūraliai konstruoja modelius, nes šių sistemų galutinis vartotojas mąsto modelio terminais ir paprastai neturi nei noro nei patirties programavimo prezentacijos detalėse.

	\item Optimizuojamumas - modeliu paremtos sistemos turi prezentacijos subsistemą, kuri gali atvaizduoti bet kokį modelį, kuris gali būti sukurtas sistemoje. Egzistuoja daug galimybių optimizacijai, nes aukšto lygio informacijos detalės yra prieinamos prezentacijos subsistemai.

	\item Reguliavimas - prezentacijos subsistema gali lengviau apibrėžti detalių išsamumo lygio valdymą bei pavyzdžių ėmimo dažnį, būtiną interaktyvioms animacijoms, remiantis reginio sudėtingumo, mašinos greičiu ir apkrova ir t.t.

	\item Mobilumas ir saugumas - modeliavimo platformos nepriklausomumas palengvina mobilių aplikacijų, kurios yra įrodytai saugios WWW(World Wide Web) programos, konstravimą.

\end{itemize}

\subsubsection{Modeliavimo esmė}

Yra keturios pagrindinės modeliavimo idėjos \cite{ElliottHudak97:Fran}:

\begin{itemize}

	\item Laikinas modeliavimas. Reikšmės, vadinamos elgesiu, kurios kinta bėgant laikui yra labiausiai dominančios. Elgesys yra pirmos klasės reikšmės ir sukurtos kompoziciškai. Lygiagretumas yra išreikštas natūraliai ir neišreikštinai. Pavyzdžiui, sekanti išraiška išreiškia animaciją (paveikslėlio elgesį), kas yra apskritimas ant kvadrato. Laiko taške t, apskritimas turi dydį sin t ir kvadratas turi dydį cos t.

\begin{lstlisting}
	bigger (sin time) circle 'over' bigger (cos time) square
\end{lstlisting}

	\item Įvykių modeliavimas. Kaip ir elgesys, įvykiai yra pirmos eilės reikšmės. Įvykiai gali reikšti tam tikrus nutikimus realiame pasaulyje (pavyzdžiui, pelės mygtuko paspaudimas) arba predikatus paremtus animacijos parametrais (pavyzdžiui, artimumą arba susidūrimą). Tokie įvykiai gali būti sujungti su kitais iki norimo sudėtingumo taip atskiriant sudėtingą animacijos logiką į semantiškai turiningus, modulius konstravimo blokus. Pavyzdžiui, įvykis, aprašantis pirmą kairio mygtuko paspaudimą po laiko t0 yra paprasčiausiai \textit{1bp t0}; aprašantis laiko kvadratą lygų penkiems yra \textit{predicate (pow(time, 2) == 5 t0)} ir jų loginė disjunkcija \textit{1bp t0 .|. predicate (pow(time, 2) == 5) t0}

	\item Deklaratyvus reaktyvumas. Elgesys dažnai yra natūraliai išreiškiamas kaip atsakas į įvykį. Bet netgi šis reaktyvus elgesys turi deklaratyvią semantiką dėl būsenos pasikeitimų, neretai įtraukiamų į įvykiais paremtą formalizmą. Pavyzdžiui, spalvos reikšmės elgesys, kuris periodiškai keičiasi iš raudonos į žalią su kiekvienu mygtuko paspaudimu gali būti aprašytas kaip paprastas pasikartojimas:

\begin{lstlisting}
	colorCycle t0 =
		red 'untilB' 1bp t0 *=> \t1 ->
		green 'untilB' 1bp t1 *=> \t2 ->
		colorCycle t2
\end{lstlisting}

	\item Polimorfinė medija. Laike besikeičiančių medijų (nuotraukos, video, garsas, 3D grafika) įvairovė ir šių tipų parametrai (erdvinės transformacijos, spalvos, taškai, vektoriai, skaičiai) turi savo pačių specialiai tipui skirtas operacijas (pavyzdžiui, nuotraukų sukimas, garso maišymas, skaitmeninė sudėtis), tačiau sutelpa į bendrinį elgesio ir reaktyvumo karkasą. Pavyzdžiui, 'untilB' operaciją naudojama prieš tai yra polimorfinė, tinkanti bet kuriai laike beisikeičiančiai reikšmei.

\end{itemize}


%%%%%%%%%%%%%%%PRIVALUMAI IR ESME END


\subsubsection{Skirtingi požiūriai}

Anot \cite{Survey}, egzistuoja du bendri požiūriai į FRP:

\begin{itemize}

	\item klasikinis FRP - elgesys ir įvykiai yra pirmos klasės ir reaktyvūs objektai;

	\item signalo-funkcijos FRP - elgesio ir įvykių transformatoriai yra pirmos eilės ir reaktyvūs objektai.

\end{itemize}

\subsubsection{Klasikinis FRP}

Anksčiausia ir dar vis standartinė FRP formuluotė \cite{ElliottHudak97:Fran} pateikia du primityvius tipų konstruktorius - Behavior (elgesys) ir Event (įvykis) - kartu su kombinatoriais, kurie pagamina šių tipų reikšmes. Lengviausias semantinis apibrėžimas šiems tipams yra pateiktas \ref{classic_semantic}.

\begin{lstlisting}[caption=- klasikinio FRP semantiniai tipai, label=classic_semantic]
	type Event a = [(Time, a)]
	type Behavior a = Time -> a
\end{lstlisting} 

Kai šie du tipų konstruktoriai yra tiesiogiai išreikšti, sistema yra žinoma kaip klasikinė FRP sistema.

\subsubsection{Klasikinio FRP istorija}

Klasikinis FRP buvo originaliai parašytas kaip Fran (Funkcininė reaktyvi animacija) \cite{ElliottHudak97:Fran}. Fran yra karkasas, skirtas deklaratyviai specifikuoti interaktyvias animacijas. Fran atvaizduoja elgesį kaip dvi funkcijas, vieną nuo laiko iki reikšmės ir kitą nuo laiko intervalo (apatinė ir viršutinė laiko riba) iki reikšmės intervalo ir naujo elgesio. Įvykiai atvaizduojami kaip ,,tobulėjančios reikšmės'', kurios imant su laiku pagamina žemesnią laiko ribą kitam atvejui, arba kitą įvykį jeigu jis iš tikrųjų įvyko.

Pirmoji FRP realizacija ne Haskell kalboje buvo Frappe \cite{Courtney:2001:FFR:645771.667929}, realizuota naudojantis Java Beans karkasu. Frappe yra sukurta remiantis įvykių supratimu bei Beans karkaso susietomis sąvybėmis (bound properties) teikiant abstrakčias sąsajas FRP įvykiams ir elgesiui bei kombinatorius kaip konkrečias klases, realizuojančias šias sąsajas. 

\subsubsection{Dabartinės klasikinės FRP sistemos}

FrTime \cite{Ignatoff:2006:CSL:2100071.2100097} kalba išplečia Scheme transliatorių nepastoviu priklausomybių grafu, kuris yra sukonstruojamas įvertinus programą. Signalų pasikeitimai atnaujina šį grafą. FrTime nesuteikia atskirų įvykių sąvokų ir pasirenka priklausomybių grafo šakas naudojant sąlyginį elgesio įvertinimą, o ne elgesio pakeitimą, naudojamą FRP sistemų.

Kaip pavyzdį galima pateikti \ref{frtime} kodo pavyzdį. Šis apibrėžia skaičialentės (tarkim Excel) tekstinį langelį, į kurį vartotojas gali įrašyti formulę. Kai vartotojas paspaudžia ant langelio, langelio adresas atsiduria įvykių sraute \textit{select-e}. Langelio pasirinkimo įvykis paveikia formulę dviem būdais. Pirma, kodas gauna pasirinkto langelio tekstą ir šis tekstas tampa formulės nauju turiniu. Antra, kodas specifikuoja, jog įvykių pasirinkimas sutelktų dėmesį į formulę \textit{formula}, taip leisdamas vartotojui redaguoti tekstą. Kai vartotojas baigia redagavimą ir paspaudžia \textit{enter} klavišą, \textit{formula} perduoda turinį išvedimo įvykių srautui. Programa apdoroja įvykį ir interpretuoja gautą tekstą (šis kodas nedemonstruojamas).

\begin{lstlisting}[caption=- skaičialentės langelio formulė (FrTime), label=frtime]
(define formula
  (new ft-text-field%
    [label ``Formula:'']
	[content-e (map-e (lambda (addr) (value-now (cell-text (addr --> key addr))))
	  select-e))]
	[focus-e select-e] ))
\end{lstlisting}

Kita nagrinėta sistema yra Reactive \cite{Elliott:2009:PFR:1596638.1596643}. Pastaroji yra dvitaktė (angl. push-pull) funkcinio reaktyvaus programavimo sistema su pirmos klasės elgesiu bei įvykiais. Pirminė Reactive įžvalga yra reaktyvumo (arba kitaip atsako pasikeitimai į įvykius, kurių atsitikimo laikas negalėjo būti žinomas prieš tai) ir laiko priklausomybės atskyrimas. Tai duoda kelią reaktyviąjai normalinėj formai, kuri atvaizduoja elgesį kaip konstantą arba paprastai priklausančią nuo laiko reikšmę, kartu su įvykių srauto nešamomis reikšmėmis, kurios irgi yra reaktyvios normalinės formos elgesys. Stūmimu (angl. push) paremtas įvertinimas yra pasiekiamas šakojant Haskell gijas, jog būtų įvertintas galvos elgesys kol yra laukiama įvykių srauto įvertinimo. Įvykus įvykiui, dabartinio elgesio gija yra nužudoma ir sukuriama nauja gija įvertinti naują elgesį. Deja Reactive realizacija naudoja nepatvarią techniką, kuri priklauso nuo gijų šakojimosi įvertinant dvi Haskell reikšmes lygiagrečiai, kad būtų galima realizuoti įvykių sujungimą. Tai priklauso nuo bibliotekos autoriaus, kad būtų galima užtikrinti darną kai naudojama ši technika ir priveda prie gijų nuotėkio kai vienas iš sujungtų įvykių yra įvykių sujungimo rezultatas.


Demonstracija pateikta \ref{Reactive} kodo pavyzdyje. Paaiškinimai:

\begin{itemize}

	\item \textit{framePass :: UI -> Event ()} yra aprūpinama duomenimis iš vartotojo sąsajos ir grąžina \textit{Event ()}, atvaizduojantį kiekvieną rėmo (angl. frame) pakitimo atsitikimą.

	\item Rezultatas paduodamas \textit{withTimeE\_ :: Ord t => EventG (Improving t) d -> EventG (Improving t) t} pakeičiant srauto () į \textit{TimeT}, kuris vaizduoja laiką kada įvyko rėmo pasikeitimas.

	\item \textit{withPrevE :: Event a -> Event (a, a)} suporuoja įvykio reikšmę su prieš tai buvusia reikšme, Tai reiškia, jog gautojo srauto pirmasis įvykis atsitinka tuo pačiu metu kaip originalaus srauto antrasis įvykis.

	\item Galiausiai apskaičiuojamas laiko \textit{(TimeT, TimeT)} skirtumas \textit{((1/) . uncurry (-))}.

\end{itemize}

\begin{lstlisting}[caption=- Reactive kodo pavyzdys, label=Reactive]
fpsE :: UI -> Event TimeT
fpsE = fmap ((1/) . uncurry (-)) . withPrevE . withTimeE_ . framePass
\end{lstlisting}

Trečia nagrinėta sistema yra Elm \cite{Czaplicki:2013:AFR:2499370.2462161}. Ją Evan Czaplicki  aprašo kaip autonominę kalbą reaktyvumui. Elm suteikia kombinatorius manipuliuoti diskrečiais įvykiais ir kompiliuojasi į Javascript kalbą, kas padaro tai naudinga kliento pusės web programavimui. Tačiau Elm nesuteikia perjungimo arba besitęsiančio laiko elgesio, nors suderinimas yra pateikiamas naudojant diskretaus laiko įvykius, kurie yra sužadinami pasikartojančiais intervalais, specifikuotais apibrėžiant įvykį. Tezė teigia, jog Arrowized FRP (signalų-funkcijų FRP) gali būti integruota į Elm, bet suteikia mažai paramos šiam teiginiui.

Kaip pavyzdį galima paanalizuoti pelės judėjimo sekimą naudojant Elm kalbą \cite{Elm}. Šis kodo pavyzdys naudoja signalus, kurie yra kertinės FRP abstrakcijos. Signalas yra reikšmė, kuri keičiasi bėgant laikui. Kodo pavyzdyje, pelės pozicija yra atvaizduojama kaip svekųjų skaičių poros signalas, fiksuojantis pelės koordinates: \textit{Mouse.position: Signal (Int, Int)}. Funkcija \textit{lift : (a -> b) -> Signal a -> Signal b} priima nuo tipo a reikšmių iki tipo b reikšmių ir pritaiko funkciją kiekvienai reikšmei, ko pasekoje yra gaunamas tipo b rezultatas. Funkcija \textit{asText : a -> Element} (paverčia Elm reikšmes į tekstinę reprezentaciją) yra pritaikoma kiekvienai \textit{Mouse.position} reikšmei taip paverčiant koordinačių signalą į \textit{Elements} tipo signalą. Kai pelės pozicija pasikeičia, taipogi pasikeičia ir rodomas elementas \textit{Element}.

\begin{lstlisting}[caption=- Pelės judėjimo sekimas (Elm), label=Elm]
	main = lift asText Mouse.position
\end{lstlisting}

Ketvirtoji nagrinėta sistema yra Reactive-banana\footnote{http://hackage.haskell.org/package/reactive-banana} biblioteka. Pastaroji yra dvitaktė (angl. push-pull) FRP sistema sukurta naudoti su Haskell GUI karkasu. Visų pirma, ji charakterizuoja monadą elgesio ir įvykių kūrimui, kuri gali būti komponuojama bei įvertinama. Ši monada apima konstrukcijas GUI bibliotekos konstrukcijų pririšimui prie primityvių įvykių. Ji privalo būti įkomponuojama į Haskell IO veiksmą įvertinimui įvykti. Reactive-banana realizacija yra panaši į FrTime - naudojant priklausomybių grafą tinklo atnaujinimui įvykus įvykiui. Reactive-banana taip pat kaip Frtime vengia apibendrinto perjungimo vietoje elgesio reikšmių šakojimosi funkcijų, bet išlaiko elgesio ir įvykių atskyrimo. Užuot apibendrinto perjungimo kombinatoriaus, kuris leidžia pakeisti sutartinį elgesį, reactive-banana suteikia žingsninį kombinatorių, kuris pažingsniui sukuria elgesį iš įvykio srauto reikšmių.

Kaip pavyzdį galima panagrinėti konkretų Reactive-banana elgesį \ref{banana}. Įprastai norint sukurtį elgesį galima naudoti dvi funkcijas \textit{stepper} ir \textit{accumB}. Abi veikia su pradine reikšme ir įvykių srautu. Skirtumas yra tas, jog \textit{stepper} pakeičia elgesio reikšmę į įvykio reikšmę, o \textit{accumB} pritaiko funkciją įvyiui, kuris yra elgesio reikšmė. Išraiška \textit{bSet = stepper 0 eNewVal} sukuria elgesį, pavadintą \textit{bSet} su pradine reikšme 0. Kai tik gaunamas įvykis iš \textit{eNewVal} srauto, \textit{bInt} reikšmė pasikeičia į įvykio reikšmę. Taigi jeigu ateina reikšmė 2, \textit{bSet} tampa lygus 2.  Kitu atveju, išraiška \textit{bUpdated = accumB 0 eUpdater} padaro \textit{bUpdated} elgesiu su pradine reikšme 0, bet ji modifikuojama su kiekvienu įvykiu. Jeigu įvykis ateina su reikšme (+1), o pradinė reikšmė yra 1, tai nauja gaunama reikšmė yra 2.

\begin{lstlisting}[caption=- Reactive-banana elgesio pavyzdys, label=banana]
eNewVal :: Event t Int
bSet :: Behavior t Int
bSet = stepper 0 eNewVal
 
eUpdater :: Event t (Int -> Int)
bUpdated :: Behavior t Int
bUpdated = accumB 0 eUpdater
\end{lstlisting}

\subsubsection{Signalo funkcijos FRP}

Alternatyvus FRP būdas pirmą kartą pasiūlytas darbe apie Fruit \cite{Courtney01genuinelyfunctional}. Fruit yra biblioteka, skirta deklaratyviam GUI specifikavimui. Biblioteka naudoja rodyklės sąvoką signalo-funkcijos abstrakcijai. Rodyklės yra abstraktaus tipo konstruktoriai su įvesties ir išvesties tipo parametrais kartu su rinkiniu maršruto parinkimo kombinatorių. Tai demonstruojama \ref{arrow_combinators}. $>>>$ operatorius naudojamas komponuoti 2 rodykles, \textit{arr} operatorius naudojamas ,,pakėlimui'', \textit{first} bei \textit{second} operatoriai, loop kombinatorius  - projekcijoms, o loop kombinatorius - ciklui, kurio pagalba galima sukurti sudėtingesnes animacijas (pavyzdžiui banguojančias).  Rodyklės idėja Haskell kalboje, įskaitant rodyklių kombinatorių aksiomas, kurias turi tenkinti, yra išvesti iš rodyklių sąvokų iš kategorijų teorijos.

\begin{lstlisting}[caption=- rodyklių kombinatoriai, label=arrow_combinators]
	(>>>) :: (Arrow a) => a b c -> a c d -> a b d
	arr :: (Arrow a) => (b -> c) -> a b c
	first :: (Arrow a) => a b c -> a (b, d) (c, d)
	second :: (Arrow a) => a b c -> a (d, b) (d, c)
	loop :: (Arrow a) => a (b, d) (c, d) => a b c
\end{lstlisting}

Signalų funkcijos yra nuo laiko priklausančios ir įvykių bei elgesio reaktyvūs transformatoriai. Elgesys ir įvykiai negali būti tiesiogiai manipuliuojami. Šis būdas turi du motyvus: padidina modalumą, kadangi signalo funkcijų įvestis ir išvestis gali būti transformuojama ir tai išvengia didelės laiko klasės ir atminties nuotėkio, kas nutinka kai FRP realizuojamas kaip pirmos klasės elgesys ir įvykiai.

Panašiai kaip ir FrTime, Netwire\footnote{http://hackage.haskell.org/ package/netwire-3.1.0} biblioteka vengia dinaminio perjungimo, šiuo atveju dėl signalo slopinimo (angl. signal inhibition, prieš išleidžiant pradinę versiją buvo tiesiog vadinama signalų intervalu). Netwire yra parašyta kaip rodyklės tranformatorius. Signalo slopinimas yra įgyvendintas padarant signalo funkcijų išvestį monoidu ir tada sujungiant signalo funkcijų išvestis. Prislopinta signalo funkcija pagamina monoido nulį (monoid's zero) kaip išvestį. Primityvai apibrėžia elgesio slopinimą ir sukomponuotos signalo funkcijos slopina jeigu jų išvestis dera su monoido nuliu.

Yampa \cite{Nilsson:2005:DOF:1090189.1086374} yra rodyklyzuotos FRP sistemos optimizacija, pirmą kartą panaudota Fruit. Yampa realizacija naudoja Generalized Algebraic Datatypes, kad leistų daug didesnę saugaus tipo duomenų tipų klasę signalo funkcijos reprezentaijai. Šis atvaizdavimas kartu su ,,išmaniais'' konstruktoriais ir kombinatoriais suteikia galimybę konstruoti rodyklizuotą FRP sistemą, kuri optimizuoja pati save. Deja pagrindinis neefektyvumas yra nereikalingi įvertinimo žingsniai dėl traukimu paremto (angl. pull-based) įvertinimo. Optimizacija yra speciali ir kieviena nauja optimizacija reikalauja naujų konstruktorių pridėjimo, taip pat kiekvieno primityvaus kombinatoriaus atnaujinimo kiekvienai konstruktorio kombinacijai palaikyti. Tačiau Yampa parodo aiškų efektyvumo privalumą lyginant su prieš tai aprašytomis rodyklizuotomis FRP realizacijomis.

Pagrindinė Yampa abstrakcija yra signalų funkcijos. Kaip pavyzdys: signalo funkcija, kuri $\alpha$ tipo signalą priskiria tipo $\beta$ signalui užrašoma SF 4alpha $\beta$ (SF $\alpha$ $\beta$ = Signal $\alpha$ -> Signal $\beta$), kur Signal $\alpha$  = Time -> $\alpha$.

Dar vienas alternatyvus FRP būdas yra N-ary FRP \cite{Sculthorpe11towardssafe}. PhD tezė  siūlo techniką tipizuojant rodyklizuotas FRP sistemas naudojant priklausomus tipus. Didžioji dalis darbo sudarė priklausomų tipų sistemos korektiškumo įrodymas. Šis darbas pristatė signalų vektorius, tipizuotą konstruktorių, kuris leidžia elgesio bei įvykių atskyrimą FRP sistemos lygyje vietoje įvykių laikymo tik specialiu elgesio tipu.

Kartais yra pravartu sekti lokalų laiką. Šios užduoties atlikimas N-Arry technika pavaizduotas \ref{n_arry} kodo pavyzdyje. Iš pradžių apibrėžiamas \textit{localTime} tipas, o laikas gali būti apskaičiuotas integruojant konstantą, lygią vienetui.

\begin{lstlisting}[caption=- N-Arry demonstracija, label=n_arry]
localTime : SF as (C Time)
localTime = constantS 1 >>> integralS
\end{lstlisting}

\subsubsection{Neįvykdyti iššūkiai}

Yra dvi pagrindinės FRP problemos. Pirma, kol signalo-funkcijos FRP yra iš prigimties saugesnė ir labiau modulinė negu klasikinė FRP, ji dar turi būti efektyviai realizuota. Klasikinės FRP programos yra pažeidžiamos dėl laiko nuotėkio bei priežastingumo pažeidimų dėl galimybės tiesiogiai manipuliuoti reaktyviomis reikšmėmis. Antra, sąsaja tarp FRP programų ir daugybės atskirų įvesties ir išvesties šaltinių išlieka specialūs ir daugeliu atveju realizacijos limituotu variantu.

Viena pagrindinė išimtis yra Reactive-banana sistema, kuri suteikia monadą primityvių įvykių konstravimui ir elgesiui iš kuriuos FRP programa gali būti sukonstruota. Tačiau šis būdas yra nelankstus, nes jis reikalauja bibliotekos palaikymo sistemai. Negana to, būnant klasikine FRP sistema, Reactive-banana pritrūksta galimybės transformuoti elgesio bei įvykių įvestį, kadangi visa įvestis yra neišreikštinė.

\subsubsection{Įvykių srautas}

Pagal \cite{Bass:2007:Mythbusters}, įvykių srautas yra eilė pagal laiką surikiuotų įvykių, pavyzdžiui akcijų rinkos srautas.

Įvykių srautas kaip duomenų srauto tipas formaliai atrodo kaip pora (s, $\Delta$), kur s yra seka surikiuotų sąrašo įvykių, o $\Delta$ yra seka laiko intervalų ir kiekvienas $\Detlta$n > 0.

Tokio duomenų srauto pavyzdžiai gali būti:

\begin{itemize}

	\item akcijų kursas,

	\item paspaudimų srautas,

	\item tinklo srautas,

	\item GPS duomenys.

\end{itemize}

Įvykių srauto apdorojimas pagal atsitikimo laiką turi privalumų:

\begin{itemize}

	\item įvykių apdorojimo algoritmai naudoja mažai atminties, nes jiems nereikia prisiminti daug įvykių;

	\item algoritmai gali būti labai greiti;

	\item gavus įvykį, skaičiavimai atliekami iškart, todėl galima perduoti rezultatą kitam skaičiavimui ir pamiršti įvykį.

\end{itemize}

Įvykių srauto apdorojimas labiau akcentuoja didelio našumo duomenų gavimą ir matematinių algoritmų pritaikymą įvykių duomenims. Taip pat įvykių srautai įprastai pritaikomi konkrečiai sistemai ar organizacijai.

\subsection{Įvykių kaupimas}

Šiame skyriuje yra aprašomos žinios apie įvykių kaupimą, pliusus ir minusus, įvykių srautus bei įvykių kaupimą funkciniame programavime remiantis daugiausia Vaughn Vernon surinkta ir aprašyta informacija \cite{vernon2013implementing}.

\subsubsection{Įvadas}

Kartais verslui svarbu fiksuoti objekto pasikeitimus domeno modelyje. Šiuos pasikeitimus galima stebėti skirtingais būdais. Įprastai yra pasirenkama stebėti kai esybė yra:

\begin{itemize}

	\item sukurta,

	\item paskutinį kartą modifikuota

	\item bei kas atliko modifikaciją.

\end{itemize}

Tačiau šis būdas nepateikia jokios informacijos apie vienkartinius pasikeitimus.

Atsiradus poreikiui stebėti pasikeitimus detaliau, verslas reikalauja dar daugiau metaduomenenų, ko pasekoje tokie faktai kaip individualios operacijos laiko tekmėje bei jų įvykdymo laikas tampa svarbūs. Šie poreikiai verčia įvesti audito žurnalą fiksuoti labai tikslias panaudojimo atvejų metrikas, tačiau pastarasis būdas turi apribojimų. Jis gali atskleisti dalį informacijos apie tai kas nutiko sistemoje, leisti rasti bei ištaisyti dalį riktų bei klaidų programinėje įrangoje. Bet audito žurnalas neleidžia patikrinti domeno objekto būsenos prieš ir po tam tikrų pasikeitimų. O jeigu būtų galima išgauti daugiau informacijos iš pasikeitimų stebėjimo?

Visi programinės įrangos kūrėjai susiduria su labai tiksliu pasikeitimų stebėjimu. Įprastas ir populiarus pavyzdys yra išeities kodo saugyklos, tokios kaip CVS\footnote{http://www.nongnu.org/cvs/}, Subversion\footnote{http://subversion.apache.org/}, Git\footnote{http://git-scm.com/} arba Mercurial\footnote{http://mercurial.selenic.com/}. Visos šios pataisų valdymo sistemos leidžia stebėti pirminių failų pasikeitimus. Įrankiai leidžia peržiūrėti išeities kodo artefaktus nuo pačios pirmosios pataisos iki paskutinės. Kai visi išeities failai yra nusiunčiami į pataisų kontrolės sistemą, ši gali stebėti pasikeitimus viso programinės įrangos kūrimo gyvavimo ciklo metu.

Jeigu šis principas būtų pritaikytas vienai esybei, tada vienam agregatui\footnote{http://martinfowler.com/bliki/DDD\_Aggregate.html} bei galiausiai kiekvienam modelio agregatui, galima suprasti kokią naudą atneša sistemos objektų pasikeitimų stebėjimas:

\begin{itemize}

	\item Kas būtent nutiko modelyje, jog agregato egzempliorius buvo sukurtas?

	\item Kas nutiko agregato egzemplioriui bėgant laikui? (Operacijų požiūriu)

\end{itemize}

Turint visų atliktų operacijų istoriją, galima palaikyti laikinus modelius. Toks kaitos stebėjimas yra įvykių kaupimo principas. \ref{pic:es} diagramoje pateikta šio principo aukšto lygio reprezentacija. Agregatai publikuoja įvykius, kurie yra išsaugomi įvykių saugykloje ir naudojami sekti modelio būsenos pasikeitimus. Verta paminėti, jog įvykiai reprezentuoja tam tikrą būsenos pasikeitimą bėgant laikui, todėl jie nėra atnaujinami arba ištrinami. Saugykla nuskaito įvykius iš įvykių saugyklos ir pritaiko juos vieną po kito taip atkurdama agregato būseną. 

\begin{figure}[ht]
	\centering
	\includegraphics[width=0.9\linewidth]{pics/es.png}
	\caption{Įvykių kaupimo aukšo lygio reprezentacija}
	\label{pic:es}
\end{figure}

\subsubsection{Įvykių kaupimo privalumai ir trūkumai}

Kaip saugojimo mechanizmas, įvykių kaupimas stipriai skiriasi ir pakeičia ORM\footnote{http://www.orm.net/} įrankį. Kadangi įvykiai dažnai įrašomi kaip dvejetainės reprezentacijos, jie negali būti optimaliai naudojami užklausoms atlikti. Faktiškai įvykių kaupimu pagrįstoms saugykloms tereikia vienos operacijos - gauti įrašus pagal unikalią agregato tapatybę \cite{CQRS:GregYoung}. To pasekoje užklausom daryti reikia kito kelio. Dažniausiai tam pasirenkamas CQRS\footnote{http://martinfowler.com/bliki/CQRS.html} principas \cite{Betts:2013:ECE:2509680}. 

Įvykių istorija gali padėti surasti bei ištaisyti sistemos defektus bei klaidas \cite{SeanFitz2012}. Derinimas naudojant istoriją visų veiksmų, kurie nutiko sistemoje, turi didžiulį pranašumą. Įvykių kaupimas gali vesti prie didelio našumo domeno modelių, tai yra palaikyti ypač didelį skaičių operacijų per sekundę. Pavyzdžiui, įrašymas į vieną duomenų saugyklos lentelę yra ypač greitas. Negana to, tai leidžia CQRS užklausų modelį išplėsti horizontaliai, nes duomenų šaltinio atnaujinimai įvykdomi fone, kai įvykių saugykla yra atnaujinama naujais įvykiais.

\subsubsection{Įvykių kaupimas funkciniame programavime}

Vaughn Vernon pateikia keletą pastebėjimų apie įvykių kaupimą funkciniame programavime, kurie gali būti naudingi atliekant projektinius sprendimus bei eksperimentinį tyrimą:

\begin{itemize}

	\item Agregatas projektuojamas kaip nekintantis būsenos įrašas kartu su funkcijomis, kurios keičia būseną. Šios funkcijos paprasčiausiai priima būsenos įrašą ir įvykių argumentus ir gražina naują būsenos įrašą kaip rezultatą. Tokia funkcija pavaizduota \ref{aggregate} kodo pavyzdyje.

\begin{lstlisting}[caption=- agregato būsenos keitimas, label=aggregate]

	Funkcija<Busena, Ivykis, Busena>

\end{lstlisting}

	\item Dabartinė agregato būsena gali būti apibrėžta kaip suskleidimas į kairę visų praeities įvykių, kurie yra perduodami būseną keičiančiai funkcijai.

	\item Agregato metodai gali būti išreikšti kaip kolekcija funkcijų be būsenos.

	\item Įvykių saugykla gali būti suvokiama bei naudojama kaip funkcinė duomenų bazė, nes ji perduoda argumentus funkcijoms, kurios keičia agregato būseną.

\end{itemize}

\subsection{Išvados}

Literatūros analizės metu remiantis kitų autorių patirtimi:

\begin{itemize}

\item išnagrinėtas funkcinis-reaktyvus programavimas,

\item išnagrinėtas įvykių kaupimo principas,

\item išnagrinėti įvykių srautai bei operacijos su jais,

\item susipažinta su įvykių kaupimu funkciniame programavime,

\item įvaldyta sąvokų sistema, susijusi su nagrinėjama tematika.

\end{itemize}

\section{Sąvokų apibrėžimai}
\begin{itemize}

	\item \textbf{Agregatas} (angl. aggregate) - DDD modelis, rinkinys domeno objektų, kurie gali būti laikomi kaip visuma.

	\item \textbf{Derinimas} (angl. debugging) - riktų bei klaidų paieška programinėje įrangoje bei jų taisymas.

	\item \textbf{Esybė} (angl. entity) - kažkas, kas egzistuoja pats savaime, faktiškai arba hipotetiškai.

	\item \textbf{Metaduomenys} (angl. metadata) - duomenys apie kitus duomenis.

	\item ą

	\item 1

\end{itemize}

\section{Santrumpos ir paaiškinimai}
\begin{itemize}

	\item \textbf{CQRS} (angl. Command Query Responsibility Segregation) – komandų-užklausų atsakomybių atskyrimas.

	\item \textbf{DDD} (angl. Domain-Driven Design) – būdas kurti programinę įrangą, skirtą spręsti sudėtingus uždavinius, bei apjungti realizaciją kartu su augančiu domeno modeliu.

	\item \textbf{NoSQL} – duomenų bazė, skirta architektūriniams modeliams, kuriems nereikia palaikyti stiprios darnos principo, kuris naudojamas reliacinėse duomenų bazėse. Tai įgalina horizontalų išplečiamumą bei aukštesnį prieinamumą.

	\item \textbf{ORM} (angl. Object-Relational Mapping) – programavimo technika duomenų konvertavimui tarp nesuderinamų sistemų tipų, naudojama objektinio programavimo kalbose. Pavyzdys: JAVA programavimo kalba bei Oracle duomenų bazė.

\end{itemize}

\bibliography{references}

\end{document}
